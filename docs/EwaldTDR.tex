\documentclass[11pt,a4paper]{article}
\usepackage[latin1]{inputenc}
\usepackage[english]{babel}
\usepackage{amsmath}
\usepackage{amsfonts}
\usepackage{amssymb}
\usepackage[left=2cm,right=2cm,top=2cm,bottom=2cm]{geometry}
\author{Sebastian Wouters}
\title{Why do we need the Ewald stuff in Waffle 0.1?}
\begin{document}
\maketitle
\section{Intro}
We need to solve the HF equations
\begin{eqnarray}
\hat{f} \psi_i(\vec{r}) & = & \left( - \frac{1}{2} \vec{\nabla}^2 - \sum\limits_{A} \frac{Z_A}{\mid \vec{r}_A - \vec{r} \mid} + \int \frac{\rho(\vec{r}')}{\mid \vec{r} - \vec{r}' \mid} d\vec{r}' \right) \psi_i(\vec{r}) - \sum\limits_j \int d\vec{r}' \left( \frac{\psi_j^*(\vec{r}') \psi_i(\vec{r}')}{\mid \vec{r}' - \vec{r} \mid} \right) \delta_{s^z_i,s^z_j} \psi_j(\vec{r})\nonumber\\
& = & \epsilon_i \psi_i(\vec{r})
\end{eqnarray}
It would be convenient to treat the direct electron-electron and electron-nuclear terms separately. We can do that by defining
\begin{eqnarray}
\tilde{\rho}_N(\vec{r}) & = & - \sum\limits_A Z_A \delta(\vec{r} - \vec{r}_A) + \frac{N}{\Omega} \\
\tilde{\rho}_e(\vec{r}) & = & \rho(\vec{r}) - \frac{N}{\Omega} \label{hhh}
\end{eqnarray}
where $\Omega$ is the size of the supercell and $\sum\limits_{\alpha} Z_{\alpha} = N$ the number of electrons in it. The total charge density is not changed by these simultaneous shifts. The shifts yield the same Fock operator and global charge distribution in space, but allow to handle the direct electron-electron term by FT and the electron-nuclear term by an Ewald summation.

\section{Direct electron-electron term}
The Poisson equation
\begin{equation}
\vec{\nabla}^2 \phi(\vec{r}) = - \tilde{\rho}_e(\vec{r})
\end{equation}
has solution
\begin{equation}
4 \pi \phi(\vec{r}) = \int \frac{\tilde{\rho}_e(\vec{r}')}{\mid \vec{r} - \vec{r}' \mid} d\vec{r}'
\end{equation}
Fourier transforming the Poisson equation yields
\begin{equation}
\vec{k}^2 \phi(\vec{k}) = \tilde{\rho}_e(\vec{k})
\end{equation}
This can be easily solved if the $\vec{k} = 0$ component of $\tilde{\rho}_e$ drops out, which we have ensured by Eq. \eqref{hhh}.

\section{The electron-nuclear term}
\begin{equation}
\int \frac{\tilde{\rho}_N(\vec{r}')}{\mid \vec{r} - \vec{r}' \mid} d\vec{r}' \label{gjgj}
\end{equation}
can be split up into
\begin{equation}
\int \frac{\tilde{\rho}_N(\vec{r}') \text{erf}\left( \frac{\sqrt{\eta}}{2} \mid \vec{r} - \vec{r}' \mid \right)}{\mid \vec{r} - \vec{r}' \mid} d\vec{r}' + \int \frac{\tilde{\rho}_N(\vec{r}') \text{erfc}\left( \frac{\sqrt{\eta}}{2} \mid \vec{r} - \vec{r}' \mid \right)}{\mid \vec{r} - \vec{r}' \mid} d\vec{r}'
\end{equation}
The complementary error function decays exponentially fast for increasing arguments, so this integral/summation can be easily done in real space. The error function part contains the long-range tail of $\frac{1}{x}$, whereas the divergence of $\frac{1}{x}$ at $x=0$ is completely contained in the complementary error function part. The erf part can hence be dealt with in reciprocal space.
\subsection{erfc}
This part is equal to
\begin{equation}
- \sum\limits_A Z_A \frac{\text{erfc}\left( \frac{\sqrt{\eta}}{2} \mid \vec{r} - \vec{r}_A \mid \right)}{\mid \vec{r} - \vec{r}_A \mid} + \frac{N}{\Omega} \int  \frac{\text{erfc}\left( \frac{\sqrt{\eta}}{2} \mid \vec{r} - \vec{r}' \mid \right)}{\mid \vec{r} - \vec{r}' \mid} d\vec{r}'
\end{equation}
The nuclear part can be rapidly truncated as the complementary error function decays exponentially with distance. The background part can be treated analytically by shifting the origin, and introducing spherical coordinates. It yields
\begin{equation}
\frac{4 \pi N}{\Omega \eta}
\end{equation}
\subsection{erf}
This part is equal to
\begin{equation}
\int \left[ - \sum\limits_{\alpha \mathbf{T}} Z_{\alpha} \delta(\vec{r}' - (\vec{r}_{\alpha} + \mathbf{T})) + \frac{N}{\Omega} \right]  \frac{\text{erf}\left( \frac{\sqrt{\eta}}{2} \mid \vec{r} - \vec{r}' \mid \right)}{\mid \vec{r} - \vec{r}' \mid} d\vec{r}'
\end{equation}
where the summation $A$ is split into the sum $\alpha$ over the charges in one unit cell and the sum $\mathbf{T}$ over the different unit cells. With
\begin{equation}
\sum\limits_{\mathbf{T}} \delta(\vec{r} - \mathbf{T}) = \frac{1}{\Omega} \sum\limits_{\mathbf{G}} e^{i \mathbf{G} . \vec{r}}
\end{equation}
with $\mathbf{G}$ the reciprocal lattice vectors, the expression reduces to
\begin{equation}
-\frac{1}{\Omega} \sum\limits_{\alpha} Z_{\alpha} \sum\limits_{\mathbf{G} \neq 0} \int e^{i \mathbf{G}.(\vec{r}' - \vec{r}_{\alpha})} \frac{\text{erf}\left( \frac{\sqrt{\eta}}{2} \mid \vec{r} - \vec{r}' \mid \right)}{\mid \vec{r} - \vec{r}' \mid} d\vec{r}'
\end{equation}
The integral can be dealt with analytically, yielding:
\begin{equation}
- \frac{4 \pi}{\Omega} \sum\limits_{\alpha} Z_{\alpha} \sum\limits_{\mathbf{G} \neq 0} \frac{e^{i \mathbf{G}.(\vec{r} - \vec{r}_{\alpha})} e^{-G^2/\eta}}{G^2}
\end{equation}
\subsection{Final expression for the electron-nuclear term}
\begin{equation}
\int \frac{\tilde{\rho}_N(\vec{r}')}{\mid \vec{r} - \vec{r}' \mid} d\vec{r}' = - \sum\limits_A Z_A \frac{\text{erfc}\left( \frac{\sqrt{\eta}}{2} \mid \vec{r} - \vec{r}_A \mid \right)}{\mid \vec{r} - \vec{r}_A \mid} + \frac{4 \pi N}{\Omega \eta} - \frac{4 \pi}{\Omega} \sum\limits_{\alpha} Z_{\alpha} \sum\limits_{\mathbf{G} \neq 0} \frac{e^{i \mathbf{G}.(\vec{r} - \vec{r}_{\alpha})} e^{-G^2/\eta}}{G^2} \label{gjgj2}
\end{equation}
All terms on the r.h.s. can be sandwiched between gaussian basis functions analytically. The first two terms are already incorporated in the one-electron integral program. For the last term, we can derive recursion relations too:
\begin{eqnarray}
(\zeta_a,\vec{n}_a+\vec{1}_i,\vec{A} \mid e^{i \vec{G}. \vec{r}} \mid \zeta_b,\vec{n}_b,\vec{B} ) & = & \left( \frac{\zeta_a A_i + \zeta_b B_i + \frac{i G_i}{2}}{\zeta_a + \zeta_b} - A_i \right) (\zeta_a,\vec{n}_a,\vec{A} \mid e^{i \vec{G}. \vec{r}} \mid \zeta_b,\vec{n}_b,\vec{B} ) \nonumber \\
& + & \frac{1}{2(\zeta_a + \zeta_b)} N_i(\vec{n}_a) (\zeta_a,\vec{n}_a-\vec{1}_i,\vec{A} \mid e^{i \vec{G}. \vec{r}} \mid \zeta_b,\vec{n}_b,\vec{B} ) \nonumber \\
& + & \frac{1}{2(\zeta_a + \zeta_b)} N_i(\vec{n}_b) (\zeta_a,\vec{n}_a,\vec{A} \mid e^{i \vec{G}. \vec{r}} \mid \zeta_b,\vec{n}_b -\vec{1}_i,\vec{B} )\\
(\zeta_a,\vec{0},\vec{A} \mid e^{i \vec{G}. \vec{r}} \mid \zeta_b,\vec{0},\vec{B} ) & = & \left( \frac{\pi}{\zeta_a + \zeta_b} \right)^{\frac{3}{2}} e^{\frac{\left( \zeta_a \vec{A} + \zeta_b \vec{B} + \frac{i \vec{G}}{2} \right)^2}{\zeta_a + \zeta_b} - (\zeta_a \vec{A}^2 + \zeta_b \vec{B}^2)}
\end{eqnarray}
These integrals can be seperated in the three cartesian coordinates. By recombining the terms corresponding to $\mathbf{G}$ and $-\mathbf{G}$, it is possible to derive recursion relations per cartesian coordinate for the cosine and sine parts of the phase operator (they mix). Together with the phase factor due to the nuclear positions, this then gives a real expression that enters the sum, which now only goes over half the reciprocal lattice vectors.
\section{The nuclear-nuclear term}
Reconsider the direct electron-electron and electron-nuclear energy. With the densities $\tilde{\rho}_e(\vec{r})$ and $\tilde{\rho}_N(\vec{r})$, the following approximation is made in the program:
\begin{equation}
\tilde{E}_{direct} + \tilde{E}_{eN} = \frac{1}{2} \int \int \frac{\rho(\vec{r}) \tilde{\rho}_e(\vec{r}')}{\mid \vec{r} - \vec{r}' \mid} d\vec{r} d\vec{r}' + \int \int \frac{\rho(\vec{r}) \tilde{\rho}_N(\vec{r}')}{\mid \vec{r} - \vec{r}' \mid} d\vec{r} d\vec{r}'
\end{equation}
Whereas for the HF equations the additional constant backgrounds cancel, this is no longer the case. The exact remaining electrostatic energy is 
\begin{equation}
\tilde{E}_{remainder} = \frac{1}{2} \int \int \frac{\rho_N(\vec{r} + \vec{\epsilon}) \rho_N(\vec{r}')}{\mid \vec{r} - \vec{r}' + \vec{\epsilon} \mid} d\vec{r} d\vec{r}' - \frac{1}{2} \sum\limits_A \frac{Z_A^2}{\mid \vec{\epsilon} \mid} - \frac{1}{2} \frac{N}{\Omega} \int \int \frac{\rho(\vec{r})}{\mid \vec{r} - \vec{r}' \mid} d\vec{r} d\vec{r}' \label{koekoek}
\end{equation}
with $\rho_N(\vec{r}) = - \sum\limits_A Z_A \delta(\vec{r} - \vec{r}_A)$ and $\vec{\epsilon}$ going to zero. In order to deal with the remaining term separately, we can make the approximation $\rho(\vec{r}) = \rho_N(\vec{r})$ in the last term, and call this approximation the nuclear-nuclear term. The nuclear-nuclear term (per supercell) is then given by
\begin{equation}
W = \sum\limits_{\alpha\beta} \frac{Z_{\alpha}Z_{\beta}}{2} \left( \frac{4\pi}{\Omega} \sum\limits_{\mathbf{G} \neq 0} \frac{e^{i \mathbf{G}.(\vec{r}_{\beta} - \vec{r}_{\alpha})} e^{-G^2/\eta}}{G^2} - \sqrt{\frac{\eta}{\pi}} \delta_{\alpha\beta} + \sum\limits_{\mathbf{T}}^{'} \frac{\text{erfc}\left( \frac{\sqrt{\eta}}{2} \mid \vec{r}_{\beta} - \vec{r}_{\alpha} + \mathbf{T} \mid \right)}{\mid \vec{r}_{\beta} - \vec{r}_{\alpha} + \mathbf{T} \mid}  \right) - \frac{4 \pi N^2}{2 \Omega \eta} \label{gjgj3}
\end{equation}
where the ' denotes that the self-interaction term is explicitly kept out for the case $\mathbf{T} = 0$.\\

Note that in Eq. \eqref{koekoek}, the divergence of the first and third terms cancel due to charge neutrality. We could therefore use this equation to calculate the exact remaining energy when the HF cycle has reached convergence (not implemented).
\end{document}