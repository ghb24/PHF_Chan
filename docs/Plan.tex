\documentclass{report}
%\documentclass[aps,preprint,nobibnotes]{revtex}
\usepackage{graphics,graphicx,amsfonts,amsmath,amsbsy,amssymb,color}
\usepackage{bm}
%\usepackage{epic}
%\usepackage{mciteplus}
\usepackage{subfigure}

\begin{document}
\title{Periodic Hartree--Fock `Hack-a-thon' planning document}
\date{\today}
\maketitle

\section{Introduction}

This is a woefully incomplete document to try to begin planning of the Periodic Hartree--Fock code (suggestions of names for the code welcome!). 
Feel free to add to this document as we go along and I will try to keep it as an up-to-date record of the status of the overall project.
The main aim is to try out a method of `crowd-coding' scientific programming which is novel, social, hopefully productive, as well as a way to learn about techniques that few of us have experience with. 
You will need to talk to others as you go along to ensure that everything fits together, so take the opportunity of us all being in the same place.
The long term goal is an open-source, ab-initio periodic Hartree--Fock package, with easy interface and clear modular structure, such that higher-level quantum
chemical methods can be added in a straightward way. Since these are likely to be the computational bottleneck ultimately, speed of the HF part is
not of paramount importance initially. For inclusivity, this will be a mixed language package (C++ and Fortran).

Github for the code is at: {\tt https://github.com/ghb24/PHF\_Chan.git}
You should branch off this code, and regularly merge from it into your own work flow to keep things up-to-date. Hopefully with the necessary modular structure, this will not
be too much of an issue.

\section{Aims}

\subsection{First stint}

Participants will need to carefully prepare their task in advance to prevent complete chaos. 
We will have short presentations on the topics that people are assigned on Friday (potentially also Saturday), 
where each person will explain what they need to do, how it fits in to the general structure of the code, 
the interface to their part, and external routines that they will need to call.

Shortly after the hackathon, we will also have debriefing short presentations, where each person presents 
their code, explains how it works, and suggests what to do next, so we can formulate a plan for the next hackathon.

\newpage

Ambition for the end of this stint is to obtain the Hartree--Fock energy for a gamma-point Hydrogen solid with minimal S-function basis, from initial guess, or if lucky, full SCF solution. 
We would ideally like to demonstrate convergence with supercell size. Initial topics may include:

\begin{itemize}
\item Basis and primitive integral evaluation -- {\em Gerald}

\item User interface, input parser and determinantion of cell and space group
\begin{itemize}
\item construction of supercell from primative cell
\end{itemize}

\item Symmetry and construction of SA-AOs

\item Coulomb lattice sums I. Expansion on grid + FFT.

\item Coulomb lattice sums II. Expand as Gaussians (Geralds idea)

\item Coulomb lattice sums III. FMM -- Fast multipole method

\item Truncated Exchange lattice sum

\item Ewald summations

\item Initial wavefunction guess

\item Fock matrix build and SCF
\begin{itemize}
\item Care needed to avoid linear dependency issues which plague crystal/cryscor
\end{itemize}

\item DFT (Gerald has code to create grids)

\item Relativity

\end{itemize}

\subsection{Topics for the future}

\begin{itemize}
\item k-point symmetry (complex integrals) and arbitrary k-point meshes
\item Gradients
\item Support for pseudopotentials
\item Extension to arbitrary basis functions
\item Band structure calculations
\item Parallelization
\item UHF
\item Molecular orbital integrals
\item Orbital localization similar to CRYSTAL?
\item Post-HF methods
\end{itemize}


\section{Coding standards}

Clear code and commenting is essential in a project with many potentially short-term developers, so others can read, understand and extend your code.
We should use doxygen to help with this. Also, to get an idea for how your code should fit into a general structure by looking at other open-source
periodic HF codes (e.g. M. Challacombe's and others? ...)

\section{Global datastructures}

We aim to set up a build environment with the main datastructures and their interfaces before Friday so you can look at them. Interfaces and classes will be simple
for ease of use and for the mixed language. Approximate initial structure will be:

\begin{itemize}
\item Primitive Unit cell:
\begin{itemize}
\item integer: nAtoms (number of atoms)
\item nAtoms $\times$ real 3-vectors: atomic coordinates
\item nAtoms $\times$ `basis' type: basis on each atom
\end{itemize}

\item Lattice:
\begin{itemize}
\item Primitive Unit cell
\item 3 $\times$ real 3-vectors: real-space lattice vectors
\item 3 $\times$ real 3-vectors: reciprocal-space lattice vectors
\item real: Volume
\item integer 3-vector: Repetition of primitive unit cell for supercell construction
\item 3 $\times$ real 3-vectors: real-space supercell vectors
\end{itemize}

\item Basis-function-group: (for given orbital `shell')
\begin{itemize}
\item integer: iCenter  (nuclear center)
\item integer: iExp     (Index into exponent array)
\item integer: nExp     (Number of exponents)
\item integer: iCon     (Index into contraction coefficient array)
\item integer: nCon     (Number of contraction coefficients)
\item integer: AngMom   (Angular momentum of shell)
\end{itemize}

\item Basis:
\begin{itemize}
\item integer: nGrp (number of basis shells)
\item integer: nBasisGrp (number of integers in `basis-function-group' - currently 6)
\item nBasisGrp $\times$ nGrp integers: Specification of basis
\item real array: Expn(*) Array of exponents indexed by iExp
\item real array: Con(*) Array of contraction coefficients indexed by iCon
\item integer: nAO Number of AOs per primitive cell
\end{itemize}
\end{itemize}
    
\subsection{Interfaces}

To do.

\section{Incomplete References}

Please feel free to add (and remove) items to this list as you find them helpful. If a reference is particularly useful, you should refer to it in the code.

\subsection{Books}

\begin{itemize}
\item C. Pisani, R. Dovesi and C. Roetti, `Hartree--Fock Ab Initio Treatment of Crystalline Systems', Lect. N. Chem. 67, Spinger Verlag, Heidelberg, (1996)

\item C. Pisani, `Quantum-Mechanical Ab-initio calculation of the Properties of Crystalline Materials',Lect. N. Chem. 67, Spinger Verlag, Heidelberg, (1996)
\end{itemize}

These are both out of the library, and on my desk.

\subsection{General periodic HF papers}
\begin{itemize}
\item R. Dovesi, B. Civalleri, R. Orlando, C. Roetti, V. R. Saunders, `Ab initio quantum simulation in solid state chemistry', Rev. Comput. Chem. 21, 1-125 (2005)
\end{itemize}
I do not seem to have access to this paper - someone let me know if they can find it

\begin{itemize}
\item Paier, J., Diaconu, CV., Scuseria, GE., Guidon, M., VandeVondele, J., Hutter, J.,`Accurate Hartree-Fock energy of extended systems using large Gaussian basis sets', PRB, 80, 174114 (2009)

\item Tymczak, CJ., Challacombe, M., `Linear scaling computation of the Fock matrix. VII. Periodic density functional theory at the Gamma point', J. Chem. Phys., 122, 134102 (2005)

\item V.R. Saunders, `Ab initio Hartree-Fock calculations for periodic systems', Faraday Symp. Chem. S. 19, 79-84 (1984)
\end{itemize}

\subsection{Gaussian basis sets for the solid state}
\begin{itemize}
\item VandeVondele, J., Hutter, J., `Gaussian basis sets for accurate calculations on molecular systems in gas and condensed phases', J. Chem. Phys., 127, 114105 (2007)
\end{itemize}

\subsection{The coulomb summation}

\begin{itemize}
\item Gan, CK., Tymczak, CJ., Challacombe, M., `Linear scaling computation of the Fock matrix. VII. Parallel computation of the Coulomb matrix', J. Chem. Phys., 121, 6608, (2004)

\item Schwegler, E., Challacombe, M., Head-Gordon, M., `A multipole acceptability criterion for electronic structure theory', J. Chem. Phys., 109, 8764 (1998)

\item Challacombe, M., White, C., HeadGordon, M., `Periodic boundary conditions and the fast multipole method', J. Chem. Phys., 107, 10131 (1997)

\item Challacombe, M., Schwegler, E., Almlof, J., `Fast assembly of the Coulomb matrix: A quantum chemical tree code', J. Chem. Phys., 104, 4685 (1996)

\item V.R. Saunders, C. Freyria Fava, R. Dovesi, L. Salasco and C. Roetti, `On the electrostatic potential in crystalline systems where the charge density is expanded in Gaussian Functions.', Mol. Phys. 77, 629-665 (1992)

\item R. Dovesi, C. Pisani, C. Roetti and V.R. Saunders, `Treatment of Coulomb interactions in Hartree-Fock calculations of periodic systems.', Phys. Rev. B 28, 5781-5792 (1983)

\end{itemize}

\subsection{The exchange series}
\begin{itemize}
\item Guidon, M, Hutter, J and VandeVondele, J, `Auxiliary Density Matrix Methods for Hartree-Fock Exchange Calculations', J. Chem. Theory Comput., 6, 2348-2364 (2010)

\item Guidon, M, Hutter, J and VandeVondele, J, `Robust Periodic Hartree-Fock Exchange for Large-Scale Simulations Using Gaussian Basis Sets', J. Chem. Theory Comput., 5, 3010-3021 (2009)

\item J. Spencer, A. Alavi, `Efficient calculation of the exact exchange energy in periodic systems using a truncated Coulomb potential', PRB, 77, 193110 (2008)

\item Weber, V., Challacombe, M., `Parallel algorithm for the computation of the Hartree-Fock exchange matrix: Gas phase and periodic parallel ONX', J. Chem. Phys., 125, 104110 (2006)

\item Tymczak, CJ., Weber, VT., Schwegler, E., Challacombe, M., `Linear scaling computation of the Fock matrix. VIII. Periodic boundaries for exact exchange at the Gamma point', J. Chem. Phys., 122, 124105 (2005)

\item Schwegler, E., Challacombe, M., `Linear scaling computation of the Fock matrix. IV. Multipole accelerated formation of the exchange matrix', J. Chem. Phys., 111, 6223 (1999)

\item Schwegler, E., Challacombe, M., `Linear scaling computation of the Hartree-Fock exchange matrix', J. Chem. Phys., 105, 2726 (1996)

\item M. Causà, R. Dovesi, R. Orlando, C. Pisani and V.R. Saunders, `Treatment of the exchange interactions in Hartree-Fock LCAO calculation of periodic systems.', J. Phys. Chem. 92, 909-913 (1988)
\end{itemize}

\subsection{Symmetry}

\begin{itemize}
\item C.M. Zicovich-Wilson and R. Dovesi, `On the use of Symmetry Adapted Crystalline Orbitals in SCF-LCAO periodic calculations. I. The construction of the Symmetrized Orbitals.', Int. J. Quantum Chem. 67, 299-309 (1998)

\item C.M. Zicovich-Wilson and R. Dovesi, `On the use of Symmetry Adapted Crystalline Orbitals in SCF-LCAO periodic calculations. II. Implementation of the Self-Consistent-Field scheme and examples.', Int. J. Quantum Chem. 67, 309-320 (1998)

\item R. Dovesi, F. Pascale and C. M. Zicovich-Wilson, `The ab initio calculation of the vibrational spectrum of crystalline compounds; the role of symmetry and related computational aspects. In: Beyond Standard Quantum Chemistry: Applications from Gas to Condensed Phases, pp. 117-138;  R Hernández-Lamoneda (Editor) Transworld Research Network, Trivandrum, Kerala, India (2007); ISBN: 978-81-7895-293-2
\end{itemize}

\subsection{Self-consistent field}
\begin{itemize}
\item VandeVondele, J., Hutter, J., `An efficient orbital transformation method for electronic structure calculations', J. Chem. Phys., 118, 4365, (2003)

\item Challacombe, M., ` A general parallel sparse-blocked matrix multiply for linear scaling SCF theory', Comput. Phys. Comm., 128, 93, (2000)

\item Challacombe, M., `A simplified density matrix minimization for linear scaling self-consistent field theory', J. Chem. Phys., 110, 2332 (1999)
\end{itemize}

\end{document}
