\documentclass{article}
\usepackage{amsmath}
\usepackage{verbatim}
%\usepackage{listings}

\begin{document}
\title{Coulomb lattice summation}
\author{}
\maketitle

We use Coulomb lattice summation to calculate the matrix elements of the
modified potential $\tilde{V}$
\begin{align}
  \tilde{V}_{\mu\nu}^{0J}
  &=\langle \mu^0|\tilde{V}|\nu^J\rangle
  =\int \mu^0(\mathbf{r}) \tilde{V}(\mathbf{r}) \nu^J(\mathbf{r}) d\mathbf{r}^3
  \\
  \tilde{V}(\mathbf{r})
  &=\int \frac{\tilde{\rho}(\mathbf{r}')}{|\mathbf{r}-\mathbf{r}'|}
  d{\mathbf{r}'}^3
  \label{}
\end{align}
where the superscripts $0$, $J$ of $\mu^0$ and $\nu^J$ labels the indices of
cells, $\nu^J$ means the $\nu$-th AO of $J$-th cell.
Therefore the matrix representation of $V$ is an array with 
$n_\text{AO} \times n_\text{AO} \times n_\text{UnitCells}$ elements.
$\tilde{V}$ is based on the screened electron density $\tilde{\rho}(\mathbf{r})$
which is the difference between the electron density $\rho(\mathbf{r})$ and
the averaged density $\rho_0$
\begin{align}
  &\tilde{\rho}(\mathbf{r})
  =\rho(\mathbf{r}) - \rho_0
  \\
  &\int_{\mathbf{r}\in C} \rho(\mathbf{r}) d\mathbf{r}^3 = N
  \\
  &\rho_0 = \frac{N}{\Omega},\quad
  \int_{\mathbf{r}\in C} \tilde{\rho}(\mathbf{r}) d\mathbf{r}^3 = 0
  \label{}
\end{align}
where the integral region $C$ is constrained to a unit cell,
$\Omega$ is the volume of a unit cell.

\section{Formulas}
Basically, the algorithm contains the following steps:
\begin{itemize}
  \item Calculate the screened density $\tilde{\rho}(\mathbf{r})$ at each mesh
    grid of real space;
  \item Use FFT to transform $\tilde{\rho}(\mathbf{r})$ to its representation
    in $\mathbf{k}$-space $\tilde{\rho}(\mathbf{k})$;
  \item Solve the Poisson equation in $\mathbf{k}$-space and get the
    $\mathbf{k}$-space potential $\tilde{V}(\mathbf{k})$;
  \item Back transform $\tilde{V}(\mathbf{k})$ to the real space
    representation $\tilde{V}(\mathbf{r})$;
  \item Calculate the matrix elements $V_{\mu\nu}^{0J}$ numerically in terms
    of the real space mesh grids.
\end{itemize}

The screened density in the real space reads
\begin{align}
  \tilde{\rho}(\mathbf{r}) = \sum_{\mu \nu IJ}D_{\mu\nu}^{IJ}
  \mu^{I}(\mathbf{r})\nu^{J}(\mathbf{r}) - \rho_0
  \label{eq:rho:r}
\end{align}
The superscripts $I$ and $J$ indicates the indices of cells.
The density matrix $D$ has the translation symmetry
\begin{equation}
  D_{\mu\nu}^{IJ} = D_{\mu\nu}^{0,J-I}
  \label{}
\end{equation}
where the superscripts $J-I$ means the cell which corresponds to the lattice
vector pointing from $I$-th cell to $J$-th cell
\begin{equation}
  \mathbf{R}^{J-I} = \mathbf{R}^J - \mathbf{R}^I
  \label{}
\end{equation}
The periodic feature basis allows us to transform the basis of $I$-th cell to
the basis of $0$-th cell
\begin{equation}
  \mu^I(\mathbf{r}) = \mu^0(\mathbf{r}-\mathbf{R}^I)
  \label{}
\end{equation}
After considering the periodic condition and the translation symmetry,
we can rewrite the expression of density Eq. \eqref{eq:rho:r} as
\begin{align}
  \tilde{\rho}(\mathbf{r})
  &=\sum_{\mu\nu IJ}D_{\mu\nu}^{I,I+J}
  \mu^I(\mathbf{r})\nu^{I+J}(\mathbf{r}) - \rho_0
  =\sum_{I} \tilde{\rho}^I(\mathbf{r})
  \label{eq:rho:div} \\
  \tilde{\rho}^I(\mathbf{r})
  &=\sum_{\mu\nu J}D_{\mu\nu}^{0J}
  \mu^I(\mathbf{r})\nu^{I+J}(\mathbf{r}) - \rho_0 \notag\\
  &=\sum_{\mu\nu J}D_{\mu\nu}^{0J}
  \mu^0(\mathbf{r}-\mathbf{R}^I)\nu^0(\mathbf{r}-\mathbf{R}^{I+J}) - \rho_0
  \label{eq:rho:ra}
\end{align}
$D_{\mu\nu}^{0J}$ yields a density matrix with
$n_\text{AO} \times n_\text{AO} \times n_\text{UnitCells}$ elements.
Eq. \eqref{eq:rho:div} decomposes the total density into the individual
cellular contributions.
Every single cell $\tilde{\rho}^I(\mathbf{r})$ contributes to the density
distribution over the whole space.
Due to the periodic condition, 
we have the same density distribution for each single cell.
So their contributions to the total density can be refined to the contribution
of $0$-th cell
\begin{align}
  \tilde{\rho}(\mathbf{r})
  &=\sum_{I}\tilde{\rho}^I(\mathbf{r})
  = \sum_{I}\tilde{\rho}^0(\mathbf{r}-\mathbf{R}^I)
  \label{eq:rho:r0} \\
  \tilde{\rho}^0(\mathbf{r})
  &=\sum_{\mu\nu J}D_{\mu\nu}^{0J}
  \mu^0(\mathbf{r})\nu^0(\mathbf{r}-\mathbf{R}^{J}) - \rho_0
\end{align}

The FFT of $\tilde{\rho}(\mathbf{k})$ in the whole real space reads
\begin{align}
  \tilde{\rho}(\mathbf{k})
  =\int d\mathbf{r}^3 \tilde{\rho}(\mathbf{r}) e^{-i\mathbf{k}\cdot\mathbf{r}}
  =\sum_{\mathbf{r}} \tilde{\rho}(\mathbf{r})
  e^{-i\mathbf{k}\cdot \mathbf{r}}
  =\sum_{\mathbf{r} I} \tilde{\rho}^I(\mathbf{r})
  e^{-i\mathbf{k}\cdot \mathbf{r}}
  \label{}
\end{align}
The subscription $\mathbf{r}$ of $\sum$ is the index of a mesh grid in
the real space.  It indicates that the integral are calculated numerically in
terms of the mesh grids.
The summation was taken over every mesh grid of the whole space.
Due to the periodic condition,
if the density distribution of one cell was calculated,
we can know the values in all the other cells.
Therefore, we can confine the FFT in whole space to the FFT in a finite space,
say the volume of one cell,
\begin{align}
  \tilde{\rho}(\mathbf{k})
  =\int_{\mathbf{r}\in C} d\mathbf{r}^3
  \tilde{\rho}(\mathbf{r}) e^{-i\mathbf{k}\cdot\mathbf{r}}
  =\sum_{\mathbf{r}\in C}
  \tilde{\rho}(\mathbf{r}) e^{-i\mathbf{k}\cdot \mathbf{r}}
  \label{}
\end{align}

The Poisson equation in real space reads
\begin{equation}
  \nabla^2 \tilde{V}(\mathbf{r}) = 4\pi \tilde{\rho}(\mathbf{r})
  \label{eq:poisson}
\end{equation}
Solving the Poisson equation in $\mathbf{k}$-space leads to
the $\mathbf{k}$-space representation of the potential
\begin{equation}
  \tilde{V}(\mathbf{k})=\frac{4\pi\tilde{\rho}(\mathbf{k})}{k^2}
  \label{}
\end{equation}
Carrying out the inverse FFT, we obtain the real space solution of the Poisson
equation \eqref{eq:poisson}
\begin{align}
  \tilde{V}(\mathbf{r})
  =\frac{1}{\Omega}\int_{\mathbf{k}\in C} d\mathbf{k}^3
  \tilde{V}(\mathbf{k}) e^{i\mathbf{k}\cdot\mathbf{r}}
  =\frac{1}{n_xn_yn_z}\sum_{\mathbf{k}\in C}
  \Phi(\mathbf{k}) e^{i\mathbf{k}\cdot \mathbf{r}}
  \label{eq:v:real}
\end{align}
where $n_x,n_y,n_z$ are the number of mesh grids along the Cartesian axes in
one cell.  We have already employed the 3D discrete Fourier transform (DFT)
and its inverse counterpart in the above equations
\begin{align}
  \tilde{\rho}(\mathbf{k}_x,\mathbf{k}_y,\mathbf{k}_z)
  &=\sum_{x=0}^{n_x-1} \sum_{y=0}^{n_y-1} \sum_{z=0}^{n_z-1}
  \tilde{\rho}(\mathbf{r}_x,\mathbf{r}_y,\mathbf{r}_z)
  \exp\left\{-2\pi i \mathbf{k}\cdot\mathbf{r}\right\},
  %\notag\\
  %&\quad \mathbf{k}_x \in [0,n_x-1],
  %\mathbf{k}_y=[0,n_y-1],\mathbf{k}_z \in 
  \\
  \tilde{\rho}(\mathbf{r}_x,\mathbf{r}_y,\mathbf{r}_z)
  &=\frac{1}{n_xn_yn_z}\sum_{x=0}^{n_x-1}
  \sum_{y=0}^{n_y-1} \sum_{z=0}^{n_z-1}
  \tilde{\rho}(\mathbf{k}_x,\mathbf{k}_y,\mathbf{k}_z)
  \exp\left\{2\pi i \mathbf{k}\cdot \mathbf{r}\right\}
  \label{}
\end{align}
Eq. \eqref{eq:v:real} gives the real space potential of one cell.
Using the translation symmetry, we can obtain the value of
$\tilde{V}(\mathbf{r})$ in the remaining cells
%\begin{equation}
%  \tilde{V}(\mathbf{r}) = \tilde{V}(\mathbf{r}-\mathbf{R}^I)
%  \label{}
%\end{equation}
\begin{equation}
  \tilde{V}(\mathbf{r}\in I) = \tilde{V}(\mathbf{r}-\mathbf{R}^I)
  \label{}
\end{equation}
Based on the periodic feature of the basis function and the potential, we
can derive the expression of $\tilde{V}_{\mu\nu}^{0J}$
\begin{align}
  V_{\mu\nu}^{0J}
  &=\sum_{\mathbf{r}} \mu^0(\mathbf{r})\tilde{V}(\mathbf{r})\nu^J(\mathbf{r})
  \\
  &=\sum_{I}\sum_{\mathbf{r}\in I} \mu^0(\mathbf{r})
  \tilde{V}(\mathbf{r} - \mathbf{R}^I)\nu^0(\mathbf{r}-\mathbf{R}^J)
  \\
  &=\sum_{I}\sum_{\mathbf{r}\in 0} \mu^0(\mathbf{r}+\mathbf{R}^I)
  \tilde{V}(\mathbf{r})\nu^0(\mathbf{r} + \mathbf{R}^{I-J})
  \label{eq:v:r0}
\end{align}

In the above equations, the summation limits of AO indices $\mu,\nu,\dots$,
cell indices $I,J,\dots$, and mesh grid indices $\mathbf{r}$ which correspond
to the electron coordinates are not yet determined.
The indices $\mu$ and $\nu$ are always limited to the number of AOs in one cell.
In principle, the cell indices $I$ should run over all unit cells in a
crystal.  The actual series of indices $I$ can always be truncated.
When we use a super cell to mimic the crystal,
we only loop over all the unit cells in the given super cell.
(??May need to shift the farthest cell $\mathbf{R}^I-\mathbf{R}^{N_{x,y,z}}$,
$N_{x,y,z}$ is the number of cells along one Cartesian axis in the super cell)
In Eq. \eqref{eq:rho:r0} and \eqref{eq:v:r0},
we constrain the electron coordinate indices $\mathbf{r}$ in the $0$-th cell.
The range of $\mathbf{r}$ thus go from 0 to the number of mesh grids in one
cell which is
\begin{equation}
  n = n_x n_y n_z
  \label{}
\end{equation}


\section{Programming model}
We set up a class ?\verb$gird_3d$ to store the
grid informations. It includes the
\begin{itemize}
  \item the coordinates of mesh grids in one cell
  \item ??the value of basis at a given grid
\end{itemize}

\subsection{API functions}
\begin{itemize}
  \item
\begin{verbatim}
void coul_matrix(lattice_info, unit_cell_info, basis_info,
        const double *den_mat, double *coul_mat);
\end{verbatim}
\begin{itemize}
  \item main function to generate the Coulomb matrix.
  \item Input: lattice info (lattice vector, electron number, ...), basis
    info., density matrix with dims nAO $\times$ nAO $\times$ nUnitCells
  \item Output: double array with dims nAO $\times$ nAO $\times$ nUnitCells
\end{itemize}

\item \verb$void init_den_grid();$

\item \verb$void del_den_grid();$

\item \verb$double density_at_grid(const double *grid_id);$
\begin{itemize}
  \item the value of density at the given grid.
\end{itemize}

\item \verb$void cells_accumulator(cells_iter, loop_terminater, );$
\begin{itemize}
  \item loop generator, to loop over cells in the super-cell.
\end{itemize}
The example below shows using \verb$cells_looper$ to find the farthest cell in
the super-cells
\end{itemize}

\section{Reference}
Mol. Phys. 77, 629

\end{document}
